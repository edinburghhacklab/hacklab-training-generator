\documentclass[a5paper]{article}
\usepackage[top=1cm,left=1cm,right=1cm,bottom=1.8cm]{geometry}
\usepackage{ltablex}
\keepXColumns
\renewcommand{\familydefault}{phv}
\begin{document}
\noindent
\textbf{Edinburgh Hacklab Training Card} \\
\\
\begin{tabularx}{\textwidth}{|l|l|X|}
\hline
\textit{Syllabus} & \textit{Version} & \textit{Trainee Name} \\
\hline
\VAR{ items.name } & \texttt{\VAR{version}} & \\
\hline
\end{tabularx}
\begin{tabularx}{\textwidth}{|l|X|\VAR{"c|" * sessions}}
    \hline
    & & \multicolumn{\VAR{sessions}}{c|}{} \\
    \textit{ID} & \textit{Topic} & \multicolumn{\VAR{sessions}}{c|}{\textit{Sessions}} \\
    \hline
    \endhead
\BLOCK{ for item in items recursive }
\BLOCK{ if item.level == 2 }
    & \multicolumn{\VAR{sessions + 1}}{c|}{} \\
    \VAR{item.section()} & \multicolumn{1}{l}{\textbf{\VAR{item.name}}} & \multicolumn{\VAR{sessions}}{c|}{} \\
    \hline
\BLOCK{ else }
    \VAR{item.section()} & \BLOCK{ if item.indent > 0 }\hspace{\VAR{item.indent * 5}mm}\BLOCK{ endif } \VAR{item.name}
    \VAR{"& " * sessions}\\ \hline
\BLOCK{ endif }
\BLOCK{ if item.children }
\VAR{ loop(item) }
\BLOCK{ endif }
\BLOCK{ endfor }
\end{tabularx}
\end{document}
